\section{Einleitung}

Die Einleitung \cite{iptables}

\subsection{Begriffsdefinitionen}

\paragraph{DMZ}



\section{Versuchsaufbau}

\subsection{Problembeschreibung}



\section{Linux Konfiguration}

\subsection{Software}

dnsutils (für nslookup)

\subsection{Netzwerkadapter}

\subsubsection{fw1: Extranet $\Longleftrightarrow$ DMZ}

Ist die Firewall zwischen Extranet $\Longleftrightarrow$ DMZ.
Die Konfigurationsdatei liegt in {\tt /etc/network/interfaces}.

\paragraph{eth0} Verbindung zur DMZ.

\begin{lstlisting}[label=lst:dmz:eth0,caption={Netzwerkadapter eth0 Konfiguration.}]
allow-hotplug eth0
iface eth0 inet static
    address 192.168.40.250
    netmask 255.255.255.0
    broadcast 192.168.40.255
    dns-search firma-a.f223
\end{lstlisting}

\paragraph{eth1} Verbindung zum Extranet bzw. Internet.

\begin{lstlisting}[label=lst:extranet:eth1,caption={Netzwerkadapter eth1 Konfiguration.}]
allow-hotplug eth1
iface eth1 inet static
    address 10.1.0.131
    netmask 255.255.255.0
    broadcast 10.1.0.0
\end{lstlisting}

\subsection{Bootskript}

Debian verwendet das sogenannte \emph{Init Script LSB}\footnote{
\url{http://wiki.debian.org/LSBInitScripts}}, welches Unterstützung
für \emph{dependency based boot sequencing}\footnote{
\url{http://wiki.debian.org/LSBInitScripts/DependencyBasedBoot}
} bietet. Das bedeutet, dass
die Programme die beim Aufstarten des Betriebssystems in einer idealen
Reihenfolge angeordnet werden, so dass es auch keine unangenehme Effekte
durch Abhängigkeiten der Programme untereinander entstehen.

Diese Bootskripte werden in {\tt /etc/init.d/} als Konfigurationsdateien
abgelegt und beginnen mit wohldefinierten Kopfzeilen, wie in Listing
\ref{lst:lsb-header} gezeigt. In diesen Zeilen können die Abhängigkeiten
konfiguriert werden.

\begin{lstlisting}[label=lst:lsb-header,caption={Init Script LSB: Kopfzeilen.}]
### BEGIN INIT INFO
# Provides:          scriptname
# Required-Start:    $remote_fs $syslog
# Required-Stop:     $remote_fs $syslog
# Default-Start:     2 3 4 5
# Default-Stop:      0 1 6
# Short-Description: Start daemon at boot time
# Description:       Enable service provided by daemon.
### END INIT INFO
\end{lstlisting}

Wenn \emph{dependency-based booting} aktiviert ist, wird ein Daemon mit seinem
Startupskript aus {\tt /etc/init.d/} mit {\tt insserv}\footnote{
\emph{Vor} Debian 6.0 wird {\tt update-rc.d} verwendet.
} gestartet:

\begin{verbatim}
insserv mydaemon_script
\end{verbatim}

Nach den Kopfzeilen kommt das eigentliche Skript, welches z.B. mittels
{\tt iptables} die Firewall konfiguriert, aber auch beliebige andere Programme
und Dienste können somit gestartet werden.

\begin{lstlisting}[label=lst:lsb-script,caption={Init Script LSB: Eigentliches Skript.}]
case "$1" in
    start)
        # Startup stuff
        echo "Daemon started."
        ;;
    stop)
        # Shutdown this service
        echo "Daemon stopped."
        ;;
    restart)
        # Restart this service
        echo "Daemon restarted."
        ;;
    *)
        # The default case
        echo "Usage $0 {start|stop|restart}"
        ;;
esac
\end{lstlisting}

\section{Firewall Konfiguration}

\subsection{fw1: Extranet $\Longleftrightarrow$ DMZ}

\subsubsection{Initialisierung}

\paragraph{Anforderung} Grundlegende Initialisierung der Firewall, d.h.
sie nimmt alle Pakete an. Zugriff auf Rechner innerhalb der DMZ ist ohne
weitere Regeln jedoch \emph{nicht} möglich, also \emph{keine} Weiterreichung von
Pakten.

\paragraph{Konfiguration} Diese Konfigurationsdatei aus Listing \ref{lst:init}
befindet sich in {\tt /etc/init.d/firewall-external.sh}.
Das Skript wird mit dem Befehl {\tt chmod u+x firewall-external.sh}
ausführbar gemacht.

\begin{lstlisting}[label=lst:init,caption={Basis Firewall Bootskript.}]
#!/bin/sh
### BEGIN INIT INFO
# Provides:          external firewall
# Required-Start:    $local_fs $remote_fs $syslog $network
# Required-Stop:     $local_fs $remote_fs $syslog $network
# Default-Start:     2 3 4 5
# Default-Stop:      0 1 6
# Short-Description: External firewall
# Description:       External firewall (sh, mb)
### END INIT INFO

case "$1" in
    start)
        # clear
        iptables -F
        iptables -t nat -F
        iptables -t mangle -F
        iptables -x

        # defaults
        iptables -P INPUT ACCEPT
        iptables -P OUTPUT ACCEPT
        iptables -P FORWARD ACCEPT

        # loopback
        iptables -A INPUT -i lo -j ACCEPT
        iptables -A OUTPUT -o lo -j ACCEPT

        # stateful rules (after that, only need to allow NEW connections)
        iptables -A INPUT -m conntrack --ctstate ESTABLISHED,RELATED -j ACCEPT
        iptables -A OUTPUT -m conntrack --ctstate ESTABLISHED,RELATED -j ACCEPT
        iptables -A FORWARD -m conntrack --ctstate ESTABLISHED,RELATED -j ACCEPT

        # drop invalid state packets
        iptables -A INPUT -m conntrack --ctstate INVALID -j DROP
        iptables -A OUTPUT -m conntrack --ctstate INVALID -j DROP
        iptables -A FORWARD -m conntrack --ctstate INVALID -j DROP

        echo "Firewall started."
        ;;

    stop)
        iptables -F
        iptables -P INPUT ACCEPT
        iptables -P OUTPUT ACCEPT

        echo "Firewall disabled."
        ;;

    restart)
        $0 stop
        sleep 1
        $0 start
        ;;

    *)
        echo "Usage $0 {start|stop|restart}"
        ;;

esac
\end{lstlisting}
%\lstinputlisting[language=Python, firstline=37, lastline=45]{source_filename.py}

\paragraph{Test} Über die virtuelle Maschine {\tt lap1.internet.f223} wird
via {\tt ping} getestet, ob eine Verbindung zwischen {\tt lap1} und {\tt fw1}
möglich ist.


\subsubsection{Weiterleitung auf Webserver}

\paragraph{Anforderung} Pakete sollten vom Extranet über die Firewall an
den Webserver {\tt srv1}, der auf Port 80 hört, weitergeleitet werden.

\paragraph{Konfiguration} Das Listing \ref{lst:webserver} wird als
Delta zum Skript aus Listing \ref{lst:init} dargestellt.
Es wird die Option {\tt DNAT} gewählt, womit das Ziel umgebogen wird.
Die Firewall sieht von Außen nun aus wie der Webserver.

\begin{lstlisting}[label=lst:webserver,caption={Webserver Forwarding.}]
#!/bin/sh
### BEGIN INIT INFO
# Provides:          external firewall
# Required-Start:    $local_fs $remote_fs $syslog $network
# Required-Stop:     $local_fs $remote_fs $syslog $network
# Default-Start:     2 3 4 5
# Default-Stop:      0 1 6
# Short-Description: External firewall
# Description:       External firewall (sh, mb)
### END INIT INFO

SRV_IP=192.168.40.1
FW2_IP=192.168.40.240
DMZ_IP=192.168.40.250
EXT_IP=10.1.0.131

forwardToSrv()
{
    PORT=$1
    iptables -t nat -A PREROUTING -p tcp -i eth1 --dport $PORT -j DNAT --to $SRV_IP
}

case "$1" in
    start)
        # clear
        iptables -F
        iptables -t nat -F
        iptables -t mangle -F
        iptables -x

        # defaults
        iptables -P INPUT ACCEPT
        iptables -P OUTPUT ACCEPT
        iptables -P FORWARD ACCEPT

        # enable Forwarding
        echo "1" > /proc/sys/net/ipv4/ip_forward

        # loopback
        iptables -A INPUT -i lo -j ACCEPT
        iptables -A OUTPUT -o lo -j ACCEPT

        # stateful rules (after that, only need to allow NEW connections)
        iptables -A INPUT -m conntrack --ctstate ESTABLISHED,RELATED -j ACCEPT
        iptables -A OUTPUT -m conntrack --ctstate ESTABLISHED,RELATED -j ACCEPT
        iptables -A FORWARD -m conntrack --ctstate ESTABLISHED,RELATED -j ACCEPT

        # drop invalid state packets
        iptables -A INPUT -m conntrack --ctstate INVALID -j DROP
        iptables -A OUTPUT -m conntrack --ctstate INVALID -j DROP
        iptables -A FORWARD -m conntrack --ctstate INVALID -j DROP

        # port Forwarding from extranet
        forwardToSrv  80  # http
        forwardToSrv 443  # https

        echo "Firewall started."
        ;;

    stop)
        iptables -F
        iptables -P INPUT ACCEPT
        iptables -P OUTPUT ACCEPT

        echo "Firewall disabled."
        ;;

    restart)
        $0 stop
        sleep 1
        $0 start
        ;;

    *)
        echo "Usage $0 {start|stop|restart}"
        ;;

esac
\end{lstlisting}

\paragraph{Test} Über den Browser des Client-Rechner {\tt lap1},
der sich im Extranet befindet.

\subsection{fw2: DMZ $\Longleftrightarrow$ Intranet}



\section{Zusammenfassung}
