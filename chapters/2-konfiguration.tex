\section{Firewall Konfiguration}

\newcommand{\lstfwa}[2]{
\lstinputlisting[
    firstline=#1,
    lastline=#2,
    firstnumber=#1
]{code/firewall-external.sh}
}

\newcommand{\lstfwb}[2]{
\lstinputlisting[
    firstline=#1,
    lastline=#2,
    firstnumber=#1
]{code/firewall-internal.sh}
}

Die Paketprüfung mit {\tt iptables} ist dreistufig aufgebaut.
Hierarchisch von oben nach unten angeordnet gibt es die Tabellen, die Chains
(Ketten) und die eigentlichen Filterregeln\footnote{
\url{http://wiki.ubuntuusers.de/iptables2}
}.
Die im Rahmen des Labors verwendeten Tabellen sind einerseits die {\tt filter}
Tabelle, welche reine Filterregeln enthält, und die {\tt nat} Tabelle, welche
für Network Address Translations und Verfahren wie Port Forwarding eingesetzt
wird.
Innerhalb jeder Tabelle existieren verschiedene Chains, welche spezifizieren,
wann ein Paket geprüft wird.
Folgende Chains existieren in den Linux {\tt iptables}:

\begin{itemize}
\item {\tt INPUT} --- betrifft Pakete, welche an einen lokalen Prozess gehen
      sollen.
\item {\tt OUTPUT} --- betrifft Pakete, welche von einem lokalen Prozess stammen.
\item {\tt FORWARD} --- betrifft Pakete, welche geroutet werden.
\item {\tt PREROUTING} --- wird auf Pakete angewendet, bevor diese geroutet
      werden (Element der {\tt nat} Tabelle).
\item {\tt POSTROUTING} --- wird auf Pakete angewendet, nachdem diese geroutet
      wurden (Element der {\tt nat} Tabelle).
\end{itemize}

\noindent Die eigentlichen Regeln werden in einer Tabelle und Chain definiert,
trifft eine Regel auf ein Paket zu, wird die in der Regel angegebene Aktion
durchgeführt.
Wenn keine Regel zutrifft, wird die allgemein gültige Policy angewendet.

\begin{figure}[h!]
  \centering
    \includegraphics[width=0.99\textwidth]{figures/iptables-filter-nat.png}
  \caption{Filter- und NAT-Tabelle von {\tt iptables}. Aus \cite{iptables} Abb. 5.13.}
  \label{fig.iptables-filter-nat}
\end{figure}

\noindent Als grundsätzliche Konfiguration beider Firewalls wird die Policy für
eingehende, ausgehende und weiterzuleitende Pakete auf {\tt DROP} gesetzt.
Dies wird mit den {\tt iptables} Befehlen aus Listing \ref{lst:policy}
realisiert.

\lstinputlisting[
    firstline=39,
    lastline=41,
    label=lst:policy,
    caption={Default-Policy der Firewalls.}
]{code/firewall-external.sh}

\noindent Da die Firewall Rechner als Router zwischen den jeweiligen Netzen
fungieren, muss grundsätzlich das Forwarding von Paketen aktiviert werden.
Für IPv4-Pakete kann dafür der folgende Befehl genutzt werden:

\begin{verbatim}
echo "1" > /proc/sys/net/ipv4/ip_forward
\end{verbatim}

\noindent Um die Kommunikation lokaler Dienste nicht zu stören, werden ein-
und ausgehende Pakete an die Loopback-Adresse ({\tt 127.0.0.1}) nicht von der
Firewall blockiert.
Für die weiterzuleitenden Pakete wird Connection Tracking
(siehe Kapitel \ref{sec.fw-tec}) verwendet.
Damit in den eigentlichen Regeln nur noch Pakete mit Status {\tt NEW} erlaubt
werden müssen, werden Pakete mit Status {\tt ESTABLISHED} oder {\tt RELATED}
grundsätzlich erlaubt.
{\tt INVALID} Pakete dagegeben werden vor allen anderen Regeln verworfen.

\lstinputlisting[
    firstline=49,
    lastline=61,
    label=lst:established_related_invalid,
    caption={Regeln für ESTABLISHED, RELATED und INVALID Pakete.}
]{code/firewall-external.sh}


\subsection{\fwa}\label{sec.konfig.fwa}

Die Konfigurationsdatei für {\tt fw1} aus Listing \ref{lst:external}
befindet sich in\\
{\tt /etc/init.d/firewall-external.sh}.
Das Skript wird mit dem Befehl {\tt chmod u+x firewall-external.sh}
ausführbar gemacht und mit {\tt insserv firewall-external.sh}
wird das Skript auch beim Aufstarten der Maschine automatisch ausgeführt,
und damit die Firewall aktiviert.

\paragraph{§ 1 Zugriff auf das Internet}

\lstfwa{46}{47}

\lstfwa{74}{79}

\paragraph{§ 2 Mail- und Webserver}

\lstfwa{23}{28}

\lstfwa{66}{72}

\paragraph{§ 3 VPN}

\lstfwa{85}{91}

\paragraph{§ 4 DNS}

\lstfwa{81}{83}

\paragraph{§ 5 Sonstiges}

\lstfwa{63}{64}

\lstfwa{93}{99}

\subsection{\fwb}

Die Konfigurationsdatei für {\tt fw2} aus Listing \ref{lst:internal}
befindet sich in\\
{\tt /etc/init.d/firewall-internal.sh} und wird mit den Befehlen
aus Kapitel \ref{sec.konfig.fwa} ausführbar gemacht.

\paragraph{§ 1 Zugriff auf das Internet}

\lstfwb{47}{48}
\lstfwb{64}{65}


\paragraph{§ 2 Mail- und Webserver}

\lstfwb{24}{29}
\lstfwb{74}{80}

\paragraph{§ 3 VPN}

\lstfwb{20}{22}
\lstfwb{84}{92}

\paragraph{§ 4 DNS}

\lstfwb{81}{82}

\paragraph{§ 5 Sonstiges}

\lstfwb{67}{72}

Protokollierung


\newpage
\section{Tests}\label{sec.tests}

\paragraph{§ 1 Zugriff auf das Internet, § 4 DNS}

Mit dem {\tt pc01.firma-a.f223} wird via Webbrowser die Internetseite
von {\tt net1.internet.f223} aufgerufen.
Damit kann gezeigt werden:
\begin{itemize}
  \item DNS-Auflösung aus § 4 der Anforderungen des Hostnames funktioniert.
  \item Ein unbeschränkter Zugriff aus dem internen Netzwerk ins Internet
        ist möglich.
  \item Das Masquerading funktioniert, da sonst keine Rückantwort möglich wäre,
        weil das interne Netzwerk aus Sicht des Internets hinter der
        öffentlichen IP-Adresse versteckt ist.
\end{itemize}

\noindent Grundsätzlich wurden alle Verbindungen mittels {\tt iptstate} analysiert.
Für den oben angegebenen Testfall zeigen die Abbildungen
\ref{fig.iptstate-extern} und \ref{fig.iptstate-intern} die existierenden
Verbindungen an.

\begin{figure}[h!]
  \centering
    \includegraphics[width=0.9\textwidth]{figures/iptstate-extern.png}
  \caption{{\tt iptstate} der externen Firewall.}
  \label{fig.iptstate-extern}
\end{figure}

\begin{figure}[h!]
  \centering
    \includegraphics[width=0.9\textwidth]{figures/iptstate-intern.png}
  \caption{{\tt iptstate} der internen Firewall.}
  \label{fig.iptstate-intern}
\end{figure}


\paragraph{§ 2 Mail- und Webserver}

{\tt pc01.firma-a.f223} aus dem internen Netzwerk hat auf
den Webserver von {\tt srv1.firma-a.f223} Zugriff.

Der externe Zugriff auf Mail- und Webserver wird mit {\tt lap01.internet.f223}
getestet.
Dies zeigt, dass das Port-Forwarding von {\tt fw1} zu {\tt srv1}
wie gewünscht funktioniert.

\paragraph{§ 3 VPN}
konnte nicht getestet werden, da auf dem neu aufgesetzten Firewall-Rechner
{\tt fw2} kein VPN-Server konfiguriert ist.


\subsection{Netzwerkkonfiguration von {\tt srv1}}

Damit beim Zugriff von einem Rechner aus dem LAN auf den Server
{\tt srv1} der \emph{Firma A} die Pakete direkt zurückgesendet werden und nicht
erst beim Default-Gateway {\tt fw1} landen, wurde auf dem
Server eine Route zum LAN mit {\tt fw2} als Gateway definiert.
Dafür wurde folgende Zeile zur Datei {\tt /etc/network/interfaces} hinzugefügt:
\begin{verbatim}
up route add -net 192.168.30.0/24 gw 192.168.40.240
\end{verbatim}


\newpage
\section{Zusammenfassung}

TODO: Zusammenfassung

