\section{Firewall Konfiguration}

\subsection{\fwa}

Ist Firewall {\tt fw1} die zwischen \fwa sitzt.


\subsubsection{Initialisierung}

\paragraph{Anforderung} Grundlegende Initialisierung der Firewall, d.h.
sie nimmt alle Pakete an. Zugriff auf Rechner innerhalb der DMZ ist ohne
weitere Regeln jedoch \emph{nicht} möglich, also \emph{keine} Weiterreichung von
Pakten.

\paragraph{Konfiguration} Diese Konfigurationsdatei aus Listing \ref{lst:init}
befindet sich in {\tt /etc/init.d/firewall-external.sh}.
Das Skript wird mit dem Befehl {\tt chmod u+x firewall-external.sh}
ausführbar gemacht. Mit {\tt insserv firewall-external.sh}
wird das Skript auch beim Aufstarten der Maschine automatisch ausgeführt,
und damit die Firewall aktiviert.

\begin{lstlisting}[label=lst:init,caption={Basis Firewall Bootskript.}]
#!/bin/sh
### BEGIN INIT INFO
# Provides:          external firewall
# Required-Start:    $local_fs $remote_fs $syslog $network
# Required-Stop:     $local_fs $remote_fs $syslog $network
# Default-Start:     2 3 4 5
# Default-Stop:      0 1 6
# Short-Description: External firewall
# Description:       External firewall (sh, mb)
### END INIT INFO

case "$1" in
    start)
        # clear
        iptables -F
        iptables -t nat -F
        iptables -t mangle -F
        iptables -x

        # defaults
        iptables -P INPUT   ACCEPT
        iptables -P OUTPUT  ACCEPT
        iptables -P FORWARD ACCEPT

        # loopback
        iptables -A INPUT  -i lo -j ACCEPT
        iptables -A OUTPUT -o lo -j ACCEPT

        # stateful rules (after that, only need to allow NEW connections)
        iptables -A INPUT   -m conntrack --ctstate ESTABLISHED,RELATED -j ACCEPT
        iptables -A OUTPUT  -m conntrack --ctstate ESTABLISHED,RELATED -j ACCEPT
        iptables -A FORWARD -m conntrack --ctstate ESTABLISHED,RELATED -j ACCEPT

        # drop invalid state packets
        iptables -A INPUT   -m conntrack --ctstate INVALID -j DROP
        iptables -A OUTPUT  -m conntrack --ctstate INVALID -j DROP
        iptables -A FORWARD -m conntrack --ctstate INVALID -j DROP

        echo "Firewall started."
        ;;

    stop)
        iptables -F
        iptables -P INPUT  ACCEPT
        iptables -P OUTPUT ACCEPT

        echo "Firewall disabled."
        ;;

    restart)
        $0 stop
        sleep 1
        $0 start
        ;;

    *)
        echo "Usage $0 {start|stop|restart}"
        ;;

esac
\end{lstlisting}
%\lstinputlisting[language=Python, firstline=37, lastline=45]{source_filename.py}

\paragraph{Test} Über die virtuelle Maschine {\tt lap1.internet.f223} wird
via {\tt ping} getestet, ob eine Verbindung zwischen {\tt lap1} und {\tt fw1}
möglich ist.


\subsubsection{Weiterleitung auf Server}

\paragraph{Anforderung} Pakete sollten vom Extranet über die Firewall an
den Web- und Mailserver {\tt srv1}, weitergeleitet werden.
Es laufen HTTP auf Port 80, HTTPS auf Port 443 und SMTP auf Port 25.

\paragraph{Konfiguration} Das Listing \ref{lst:webserver} ist das Skript
aus Listing \ref{lst:init}, entsprechend erweitert.
Es wird die Option {\tt DNAT} gewählt, womit das Ziel umgebogen wird.
Die Firewall sieht von Außen nun aus wie der jeweilige Server.

\begin{lstlisting}[label=lst:webserver,caption={Web- und Mailserver Forwarding.}]
#!/bin/sh
### BEGIN INIT INFO
# Provides:          external firewall
# Required-Start:    $local_fs $remote_fs $syslog $network
# Required-Stop:     $local_fs $remote_fs $syslog $network
# Default-Start:     2 3 4 5
# Default-Stop:      0 1 6
# Short-Description: External firewall
# Description:       External firewall (sh, mb)
### END INIT INFO

SRV_IP=192.168.40.1
FW2_IP=192.168.40.240
DMZ_IP=192.168.40.250
EXT_IP=10.1.0.131

forwardToSrv()
{
    PORT=$1
    iptables -t nat -A PREROUTING -p tcp -i eth1 --dport $PORT -j DNAT --to $SRV_IP
}

case "$1" in
    start)
        # clear
        iptables -F
        iptables -t nat -F
        iptables -t mangle -F
        iptables -x

        # defaults
        iptables -P INPUT   ACCEPT
        iptables -P OUTPUT  ACCEPT
        iptables -P FORWARD ACCEPT

        # enable Forwarding
        echo "1" > /proc/sys/net/ipv4/ip_forward

        # loopback
        iptables -A INPUT  -i lo -j ACCEPT
        iptables -A OUTPUT -o lo -j ACCEPT

        # stateful rules (after that, only need to allow NEW connections)
        iptables -A INPUT   -m conntrack --ctstate ESTABLISHED,RELATED -j ACCEPT
        iptables -A OUTPUT  -m conntrack --ctstate ESTABLISHED,RELATED -j ACCEPT
        iptables -A FORWARD -m conntrack --ctstate ESTABLISHED,RELATED -j ACCEPT

        # drop invalid state packets
        iptables -A INPUT   -m conntrack --ctstate INVALID -j DROP
        iptables -A OUTPUT  -m conntrack --ctstate INVALID -j DROP
        iptables -A FORWARD -m conntrack --ctstate INVALID -j DROP

        # port Forwarding from extranet
        forwardToSrv  25  # smtp
        forwardToSrv  80  # http
        forwardToSrv 443  # https

        echo "Firewall started."
        ;;

    stop)
        iptables -F
        iptables -P INPUT  ACCEPT
        iptables -P OUTPUT ACCEPT

        echo "Firewall disabled."
        ;;

    restart)
        $0 stop
        sleep 1
        $0 start
        ;;

    *)
        echo "Usage $0 {start|stop|restart}"
        ;;

esac
\end{lstlisting}

\paragraph{Test} Über den Browser des Client-Rechner {\tt lap1},
der sich im Extranet befindet.


\subsubsection{Masquerading}

\paragraph{Anforderung}
Dabei soll {\tt pc01} nach außen hin wirken wie wenn die Anfrage von
{\tt fw1} kommt.

Hier wird also der Rückkanal von {\tt fw1} zu {\tt pc01} gebildet.

\paragraph{Konfiguration}

\begin{lstlisting}[label=lst:masq,caption={Masquerading.}]
#!/bin/sh
### BEGIN INIT INFO
# Provides:          external firewall
# Required-Start:    $local_fs $remote_fs $syslog $network
# Required-Stop:     $local_fs $remote_fs $syslog $network
# Default-Start:     2 3 4 5
# Default-Stop:      0 1 6
# Short-Description: External firewall
# Description:       External firewall (sh, mb)
### END INIT INFO

SRV_IP=192.168.40.1
FW2_IP=192.168.40.240
DMZ_IP=192.168.40.250
DMZ_DEV="eth0"
EXT_IP=10.1.0.131
EXT_DEV="eth1"

forwardToSrv()
{
    PORT=$1
    iptables -t nat -A PREROUTING -p tcp -i $EXT_DEV --dport $PORT -j DNAT --to $SRV_IP
}

case "$1" in
    start)
        # clear
        iptables -F
        iptables -t nat -F
        iptables -t mangle -F
        iptables -x

        # defaults
        iptables -P INPUT   ACCEPT
        iptables -P OUTPUT  ACCEPT
        iptables -P FORWARD ACCEPT

        # enable Forwarding
        echo "1" > /proc/sys/net/ipv4/ip_forward

        # fw2 as gateway to lan
        route add -net 192.168.30.0 netmask 255.255.255.0 gw $FW2_IP

        # loopback
        iptables -A INPUT  -i lo -j ACCEPT
        iptables -A OUTPUT -o lo -j ACCEPT

        # stateful rules (after that, only need to allow NEW connections)
        iptables -A INPUT   -m conntrack --ctstate ESTABLISHED,RELATED -j ACCEPT
        iptables -A OUTPUT  -m conntrack --ctstate ESTABLISHED,RELATED -j ACCEPT
        iptables -A FORWARD -m conntrack --ctstate ESTABLISHED,RELATED -j ACCEPT

        # drop invalid state packets
        iptables -A INPUT   -m conntrack --ctstate INVALID -j DROP
        iptables -A OUTPUT  -m conntrack --ctstate INVALID -j DROP
        iptables -A FORWARD -m conntrack --ctstate INVALID -j DROP

        # port Forwarding from extranet
        forwardToSrv  25  # smtp
        forwardToSrv  80  # http
        forwardToSrv 443  # https

        # masquerading for packets to extranet
        iptables -t nat -A POSTROUTING -o $EXT_DEV -s $SRV_IP          -j MASQUERADE
        iptables -t nat -A POSTROUTING -o $EXT_DEV -s $192.168.30.0/24 -j MASQUERADE

        # allow forwarding from lan to extranet
        iptables -A FORWARD -i $DMZ_DEV -o $EXT_DEV -s 192.168.30.0/24 -d 10.1.0.0/24 -m conntrack --ctstate NEW -j ACCEPT

        echo "Firewall started."
        ;;

    stop)
        iptables -F
        iptables -P INPUT  ACCEPT
        iptables -P OUTPUT ACCEPT

        echo "Firewall disabled."
        ;;

    restart)
        $0 stop
        sleep 1
        $0 start
        ;;

    *)
        echo "Usage $0 {start|stop|restart}"
        ;;

esac
\end{lstlisting}


\subsubsection{Finish}

\begin{lstlisting}[label=lst:masq,caption={Masquerading.}]
#!/bin/sh
### BEGIN INIT INFO
# Provides:          external firewall
# Required-Start:    $local_fs $remote_fs $syslog $network
# Required-Stop:     $local_fs $remote_fs $syslog $network
# Default-Start:     2 3 4 5
# Default-Stop:      0 1 6
# Short-Description: External firewall
# Description:       External firewall (sh, mb)
### END INIT INFO

SRV_IP=192.168.40.1
FW2_IP=192.168.40.240
DMZ_IP=192.168.40.250
DMZ_DEV="eth0"
EXT_IP=10.1.0.131
EXT_DNS=10.1.0.1
EXT_DEV="eth1"
EXT_NET=10.1.0.0/24
LAN_NET=192.168.30.0/24

forwardToSrv()
{
    PORT=$1
    iptables -t nat -A PREROUTING -i $EXT_DEV -p tcp --dport $PORT -j DNAT --to $SRV_IP
    iptables -A FORWARD -i $EXT_DEV -o $DMZ_DEV -d $SRV_IP -p tcp --dport $PORT -m conntrack --ctstate NEW -j ACCEPT
}

case "$1" in
    start)
        # clear
        iptables -F
        iptables -t nat -F
        iptables -t mangle -F
        iptables -x

        # defaults
        iptables -P INPUT   DROP
        iptables -P OUTPUT  DROP
        iptables -P FORWARD DROP

        # enable Forwarding
        echo "1" > /proc/sys/net/ipv4/ip_forward

        # fw2 as gateway to lan
        route add -net $LAN_NET gw $FW2_IP

        # loopback
        iptables -A INPUT  -i lo -j ACCEPT
        iptables -A OUTPUT -o lo -j ACCEPT

        # stateful rules (after that, only need to allow NEW connections)
        iptables -A INPUT   -m conntrack --ctstate ESTABLISHED,RELATED -j ACCEPT
        iptables -A OUTPUT  -m conntrack --ctstate ESTABLISHED,RELATED -j ACCEPT
        iptables -A FORWARD -m conntrack --ctstate ESTABLISHED,RELATED -j ACCEPT

        # drop invalid state packets
        iptables -A INPUT   -m conntrack --ctstate INVALID -j DROP
        iptables -A OUTPUT  -m conntrack --ctstate INVALID -j DROP
        iptables -A FORWARD -m conntrack --ctstate INVALID -j DROP

        # enable ping to this firewall server
        iptables -A INPUT -p icmp -m icmp --icmp-type 8 -m conntrack --ctstate NEW -j ACCEPT

        # port Forwarding from extranet
        forwardToSrv  25  # smtp
        forwardToSrv  80  # http
        forwardToSrv 443  # https

        # masquerading for packets to extranet
        iptables -t nat -A POSTROUTING -o $EXT_DEV -s $SRV_IP  -j MASQUERADE
        iptables -t nat -A POSTROUTING -o $EXT_DEV -s $LAN_NET -j MASQUERADE

        # allow forwarding from lan to extranet
        iptables -A FORWARD -i $DMZ_DEV -o $EXT_DEV -s $LAN_NET -d $EXT_NET -m conntrack --ctstate NEW -j ACCEPT

        # allow dns from srv to extranet
        iptables -A FORWARD -i $DMZ_DEV -o $EXT_DEV -s $SRV_IP -d $EXT_DNS -p tcp --dport 53 -m conntrack --ctstate NEW -j ACCEPT
        iptables -A FORWARD -i $DMZ_DEV -o $EXT_DEV -s $SRV_IP -d $EXT_DNS -p udp --dport 53 -m conntrack --ctstate NEW -j ACCEPT

        # protocol all other requests from extranet
        iptables -A FORWARD -i $EXT_DEV -j LOG --log-prefix "from extranet - forbidden: "
        iptables -A FORWARD -i $EXT_DEV -j DROP

        # protocol all other requests from dmz
        iptables -A FORWARD -i $DMZ_DEV -j LOG --log-prefix "from dmz - forbidden: "
        iptables -A FORWARD -i $DMZ_DEV -j DROP

        echo "Firewall started."
        ;;

    stop)
        iptables -F
        iptables -P INPUT  ACCEPT
        iptables -P OUTPUT ACCEPT

        route del -net $LAN_NET

        echo "Firewall disabled."
        ;;

    restart)
        $0 stop
        sleep 1
        $0 start
        ;;

    *)
        echo "Usage $0 {start|stop|restart}"
        ;;

esac
\end{lstlisting}


\subsection{\fwb}

Ist Firewall {\tt fw2} die zwischen \fwb sitzt.

\subsubsection{Verbindung von pc01 nach inet1}

\paragraph{Anforderung} Es soll eine Verbindung vom {\tt pc01} nach
außen ins Extranet, also auf {\tt inet1}, möglich sein.


\paragraph{Konfiguration}
% Damit {\tt pc01} nach außen hin wie die externe Firewall
% {\tt fw1} aussieht, muss die Option MASQUERADING genutzt werden.

Da {\tt fw2} nur weiterleitet, wird bei den stateful
Regeln nur FORWARD benötigt.

Es muss noch eine Route gesetzt werden, damit {\tt pc01} auch das Extranet
auffinden kann.

\begin{lstlisting}[label=lst:masq,caption={Basisskript interne Firewall.}]
#!/bin/sh
### BEGIN INIT INFO
# Provides:          internal firewall
# Required-Start:    $local_fs $remote_fs $syslog $network
# Required-Stop:     $local_fs $remote_fs $syslog $network
# Default-Start:     2 3 4 5
# Default-Stop:      0 1 6
# Short-Description: Internal firewall
# Description:       Internal firewall (sh, mb)
### END INIT INFO

SRV_IP=192.168.40.1
FW1_IP=192.168.40.250
DMZ_IP=192.168.40.240
DMZ_DEV="eth1"
LAN_IP=192.168.30.240
LAN_DEV="eth0"

forwardToSrv()
{
    PORT=$1
    TYPE=$2
    iptables -A FORWARD -i $LAN_DEV -o $DMZ_DEV -d $SRV_IP -p $TYPE --dport $PORT -m conntrack --ctstate NEW -j ACCEPT
}

case "$1" in
    start)
        # clear
        iptables -F
        iptables -t nat -F
        iptables -t mangle -F
        iptables -x

        # defaults
        iptables -P INPUT   ACCEPT
        iptables -P OUTPUT  ACCEPT
        iptables -P FORWARD ACCEPT

        # enable Forwarding
        echo "1" > /proc/sys/net/ipv4/ip_forward

        # fw1 as gateway to extranet
        route add -net 10.1.0.0 netmask 255.255.255.0 gw $FW1_IP

        # loopback
        iptables -A INPUT  -i lo -j ACCEPT
        iptables -A OUTPUT -o lo -j ACCEPT

        # stateful rules (after that, only need to allow NEW connections)
        iptables -A FORWARD -m conntrack --ctstate ESTABLISHED,RELATED -j ACCEPT

        # drop invalid state packets
        iptables -A FORWARD -m conntrack --ctstate INVALID -j DROP

        # access from lan to extranet
        iptables -A FORWARD -i $LAN_DEV -o DMZ_DEV -d 10.1.0.0/24 -m conntrack --ctstate NEW -j ACCEPT

        # access from lan to server
        forwardToSrv  80 tcp  # http
        forwardToSrv 443 tcp  # https
        forwardToSrv  25 tcp  # smtp
        forwardToSrv  53 tcp  # dns
        forwardToSrv  53 udp  # dns

        # protocol all other requests
        iptables -A FORWARD -i $LAN_DEV -j LOG --log-prefix "from lan - forbidden: "
        iptables -A FORWARD -i $LAN_DEV -j REJECT

        # protocol requests from DMZ
        iptables -A FORWARD -i $DMZ_DEV -j LOG --log-prefix "from dmz - forbidden: "
        iptables -A FORWARD -i $DMZ_DEV -j REJECT

        echo "Firewall started."
        ;;

    stop)
        iptables -F
        iptables -P INPUT  ACCEPT
        iptables -P OUTPUT ACCEPT

        echo "Firewall disabled."
        ;;

    restart)
        $0 stop
        sleep 1
        $0 start
        ;;

    *)
        echo "Usage $0 {start|stop|restart}"
        ;;

esac
\end{lstlisting}

\paragraph{Test} Funktioniert nur bei aktiven Masquerading bei {\tt fw1}.
Dazu wird von {\tt pc01} auf den Webserver von {\tt net1} zugegriffen.


\subsubsection{Fertig}

\begin{lstlisting}[label=lst:masq,caption={Basisskript interne Firewall.}]
#!/bin/sh
### BEGIN INIT INFO
# Provides:          internal firewall
# Required-Start:    $local_fs $remote_fs $syslog $network
# Required-Stop:     $local_fs $remote_fs $syslog $network
# Default-Start:     2 3 4 5
# Default-Stop:      0 1 6
# Short-Description: Internal firewall
# Description:       Internal firewall (sh, mb)
### END INIT INFO

SRV_IP=192.168.40.1
FW1_IP=192.168.40.250
DMZ_IP=192.168.40.240
DMZ_DEV="eth1"
LAN_IP=192.168.30.240
LAN_DEV="eth0"
EXT_NET=10.1.0.0/24

forwardToSrv()
{
    PORT=$1
    TYPE=$2
    iptables -A FORWARD -i $LAN_DEV -o $DMZ_DEV -d $SRV_IP -p $TYPE --dport $PORT -m conntrack --ctstate NEW -j ACCEPT
}

case "$1" in
    start)
        # clear
        iptables -F
        iptables -t nat -F
        iptables -t mangle -F
        iptables -x

        # defaults
        iptables -P INPUT   DROP
        iptables -P OUTPUT  DROP
        iptables -P FORWARD DROP

        # enable Forwarding
        echo "1" > /proc/sys/net/ipv4/ip_forward

        # fw1 as gateway to extranet
        route add -net $EXT_NET gw $FW1_IP

        # loopback
        iptables -A INPUT  -i lo -j ACCEPT
        iptables -A OUTPUT -o lo -j ACCEPT

        # stateful rules (after that, only need to allow NEW connections)
        iptables -A INPUT   -m conntrack --ctstate ESTABLISHED,RELATED -j ACCEPT
        iptables -A OUTPUT  -m conntrack --ctstate ESTABLISHED,RELATED -j ACCEPT
        iptables -A FORWARD -m conntrack --ctstate ESTABLISHED,RELATED -j ACCEPT

        # drop invalid state packets
        iptables -A INPUT   -m conntrack --ctstate INVALID -j DROP
        iptables -A OUTPUT  -m conntrack --ctstate INVALID -j DROP
        iptables -A FORWARD -m conntrack --ctstate INVALID -j DROP

        # access from lan to extranet
        iptables -A FORWARD -i $LAN_DEV -o DMZ_DEV -d $EXT_NET -m conntrack --ctstate NEW -j ACCEPT

        # enable ping to this firewall server
        iptables -A INPUT -p icmp -m icmp --icmp-type 8 -m conntrack --ctstate NEW -j ACCEPT

        # enable ping to fw1 and srv1
        iptables -A FORWARD -i $LAN_DEV -o $DMZ_DEV -d $FW1_IP -p icmp -m icmp --icmp-type 8 -m conntrack --ctstate NEW -j ACCEPT
        iptables -A FORWARD -i $LAN_DEV -o $DMZ_DEV -d $SRV_IP -p icmp -m icmp --icmp-type 8 -m conntrack --ctstate NEW -j ACCEPT

        # access from lan to server
        forwardToSrv  80 tcp  # http
        forwardToSrv 443 tcp  # https
        forwardToSrv  25 tcp  # smtp
        forwardToSrv  53 tcp  # dns
        forwardToSrv  53 udp  # dns

        # protocol all other requests
        iptables -A FORWARD -i $LAN_DEV -j LOG --log-prefix "from lan - forbidden: "
        iptables -A FORWARD -i $LAN_DEV -j DROP

        # protocol requests from DMZ
        iptables -A FORWARD -i $DMZ_DEV -j LOG --log-prefix "from dmz - forbidden: "
        iptables -A FORWARD -i $DMZ_DEV -j DROP

        echo "Firewall started."
        ;;

    stop)
        iptables -F
        iptables -P INPUT  ACCEPT
        iptables -P OUTPUT ACCEPT

        route del -net $EXT_NET

        echo "Firewall disabled."
        ;;

    restart)
        $0 stop
        sleep 1
        $0 start
        ;;

    *)
        echo "Usage $0 {start|stop|restart}"
        ;;

esac
\end{lstlisting}
\section{Zusammenfassung}



\begin{appendix}
\section{\fwa}

\lstinputlisting{code/firewall-external.sh}
%\lstinputlisting[firstline=37, lastline=45]{source_filename.py}
\end{appendix}
