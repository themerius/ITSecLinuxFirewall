\section{Firewall Konfiguration}

\subsection{\fwa}

Ist Firewall {\tt fw1} die zwischen \fwa sitzt.


\subsubsection{Initialisierung}

\paragraph{Anforderung} Grundlegende Initialisierung der Firewall, d.h.
sie nimmt alle Pakete an. Zugriff auf Rechner innerhalb der DMZ ist ohne
weitere Regeln jedoch \emph{nicht} möglich, also \emph{keine} Weiterreichung von
Pakten.

\paragraph{Konfiguration} Diese Konfigurationsdatei aus Listing \ref{lst:init}
befindet sich in {\tt /etc/init.d/firewall-external.sh}.
Das Skript wird mit dem Befehl {\tt chmod u+x firewall-external.sh}
ausführbar gemacht. Mit {\tt insserv firewall-external.sh}
wird das Skript auch beim Aufstarten der Maschine automatisch ausgeführt,
und damit die Firewall aktiviert.

\lstinputlisting[
    firstline=23,
    lastline=28,
    label=lst:forwardToSrv,
    caption={Forwarding zu Ports als Funktion.}
]{code/firewall-external.sh}

\paragraph{Test} Über die virtuelle Maschine {\tt lap1.internet.f223} wird
via {\tt ping} getestet, ob eine Verbindung zwischen {\tt lap1} und {\tt fw1}
möglich ist.


\subsubsection{Weiterleitung auf Server}

\paragraph{Anforderung} Pakete sollten vom Extranet über die Firewall an
den Web- und Mailserver {\tt srv1}, weitergeleitet werden.
Es laufen HTTP auf Port 80, HTTPS auf Port 443 und SMTP auf Port 25.

\paragraph{Konfiguration} Das Listing \ref{lst:webserver} ist das Skript
aus Listing \ref{lst:init}, entsprechend erweitert.
Es wird die Option {\tt DNAT} gewählt, womit das Ziel umgebogen wird.
Die Firewall sieht von Außen nun aus wie der jeweilige Server.


\paragraph{Test} Über den Browser des Client-Rechner {\tt lap1},
der sich im Extranet befindet.


\subsubsection{Masquerading}

\paragraph{Anforderung}
Dabei soll {\tt pc01} nach außen hin wirken wie wenn die Anfrage von
{\tt fw1} kommt.

Hier wird also der Rückkanal von {\tt fw1} zu {\tt pc01} gebildet.

\paragraph{Konfiguration}


\subsection{\fwb}

Ist Firewall {\tt fw2} die zwischen \fwb sitzt.

\subsubsection{Verbindung von pc01 nach inet1}

\paragraph{Anforderung} Es soll eine Verbindung vom {\tt pc01} nach
außen ins Extranet, also auf {\tt inet1}, möglich sein.


\paragraph{Konfiguration}
% Damit {\tt pc01} nach außen hin wie die externe Firewall
% {\tt fw1} aussieht, muss die Option MASQUERADING genutzt werden.

Da {\tt fw2} nur weiterleitet, wird bei den stateful
Regeln nur FORWARD benötigt.

Es muss noch eine Route gesetzt werden, damit {\tt pc01} auch das Extranet
auffinden kann.

\paragraph{Test} Funktioniert nur bei aktiven Masquerading bei {\tt fw1}.
Dazu wird von {\tt pc01} auf den Webserver von {\tt net1} zugegriffen.


\section{Zusammenfassung}

Zusammenfassung
